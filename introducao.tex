\section{Introdução}

\subsection{Caracterização da aplicação}
	\begin{frame}{\insertsubsection}
	\fig{wanke3}{Indústria de eletrodomésticos com extrusoras de plástico.}
	\end{frame}

	\begin{frame}{\insertsubsection}
	\fig{wanke2}{Moldes das extrusoras precisam ser substituidos através de uma ponte rolante.}
	\end{frame}

	\begin{frame}{\insertsubsection}
	\fig{wanke1}{Estrutura da ponte rolante pela fábrica.}
	\end{frame}

	\begin{frame}{\insertsubsection}
		\textbf{Sistema de segurança}
		\begin{itemize}
			\item Detectar pessoas automaticamente na região de trabalho.
			\item Impedir movimentação da ponte ao detectar pessoas.
		\end{itemize}
	\end{frame}

\subsection{Aprendizado de máquina}
	\begin{frame}{\insertsubsection}
	``Aprendizado de máquina é o campo de estudo que dá ao computador a habilidade de desempenhar uma função sem ser explicitamente programado para fazer isso.''

	\textbf{Arthur Samuel}
	\end{frame}

	\begin{frame}{Histórico}
		\begin{description}
			\item [1952] Samuel cria primeiro programa com aprendizado: damas.
			\item [1957] Perceptron de Rosenblatt: modelo biológico.
			\item [1970] AI Winter: o ceticismo depois da euforia.
			\item [1986] Processo de aprendizado \emph{backpropagation} é redescoberto.
			\item [1995] Técnica de \emph{Support Vector Machine} se tornara clássica.
			\item [2006] Avanços computacionais permitem a criação de redes neurais profundas.
		\end{description}
	\end{frame}

	\begin{frame}{Comparativo métodos de aprendizado}
	%https://www.quora.com/What-is-the-difference-between-deep-learning-and-shallow-learning#
	\begin{columns}[T]
		\column{0.5\textwidth}
		\textbf{Superficiais}
		\begin{itemize}
			\item Requer pré-processamento das amostras: descrição.
			\item Criar um extrator de características requer conhecimento do problema.
			\item A descrição da amostra é crítica, portanto tem mais ênfase que o modelo em si.
		\end{itemize}
		\vfill

		\column{0.5\textwidth}
		\textbf{Profundos}
		\begin{itemize}
			\item Tem como entrada dados ``crus''.
			\item Aprendem a melhor representação através de otimização -- aprendizado de representação.
			\item Requer um grande número de amostras para o treinamento.
		\end{itemize}
		\vfill
	\end{columns}
	\end{frame}

\subsection{Método clássico de detecção de objetos}
	\begin{frame}{\insertsubsection}
	\begin{enumerate}
	\item Identificar uma região de interesse na imagem.
	\item Utilizar um extrator de características para descrição do objeto.
	\item Introduzir a amostra, proveniente do descritor, em um classificador simples, obtendo a classe correspondente.
	\end{enumerate}

	\fig{obj-detection}{Processo de detecção de objetos.}
	\end{frame}

	\begin{frame}{Panorama do trabalho}
		\tableofcontents
	\end{frame}
