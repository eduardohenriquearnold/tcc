\section{Introdução}

\subsection{Caracterização da aplicação}
	\begin{frame}{\insertsubsection}
	\fig{wanke3}{Indústria de eletrodomésticos com extrusoras de plástico.}
	\end{frame}

	\begin{frame}{\insertsubsection}
	\fig{wanke2}{Moldes das extrusoras precisam ser substituidos através de uma ponte rolante.}
	\end{frame}

	\begin{frame}{\insertsubsection}
	\fig{wanke1}{Estrutura da ponte rolante pela fábrica.}
	\end{frame}

	\begin{frame}{Sistema de segurança}
		\textbf{Objetivos} \\
		\begin{itemize}
			\item Detectar pessoas automaticamente na região de trabalho.
			\item Impedir movimentação da ponte ao detectar pessoas.
		\end{itemize}

		\pause

		\textbf{Funcionamento} \\
		\begin{itemize}
			\item Câmera de profundidade com vista superior da área de trabalho.
			\item Aprendizado de máquina e visão computacional para detecção de pessoas.
		\end{itemize}
	\end{frame}

\subsection{Aprendizado de máquina}
%https://www.quora.com/What-is-the-difference-between-deep-learning-and-shallow-learning#
%http://sebastianraschka.com/faq/docs/difference-deep-and-normal-learning.html

	\begin{frame}{Métodos tradicionais de aprendizado}
		\begin{itemize}
			\item Requerem pré-processamento das amostras: descrição.
			\item Criar um extrator de características requer conhecimento do problema.
			\item A descrição da amostra é crítica, em geral tem mais ênfase que o modelo em si.
			\item Modelos comuns: SVM, redes neurais, \emph{random forests}.
		\end{itemize}
	\end{frame}
	
	\begin{frame}{Métodos de aprendizado profundo}
		\begin{itemize}
			\item Tem como entrada dados ``crus''.
			\item Aprendizado de representação.
			\item Modelo com muitos parâmetros e diversas camadas.
			\item Requer um grande número de amostras para o treinamento.
			\item Inovações que permitem treinamento.
			\item Modelos comuns: Redes neurais profundas, convolucionais, recursivas (LSTM).
		\end{itemize}
	\end{frame}

\subsection{Método de detecção de objetos}
	\begin{frame}{Utilizando métodos tradicionais}
	\begin{enumerate}
	\item Identificar candidatos na imagem.
	\item Utilizar um extrator de características para descrição do objeto.
	\item Introduzir a amostra, proveniente do descritor, em um classificador SVM, obtendo a classe correspondente.
	\end{enumerate}

	\fig{diagram/system}{}[1.1][1]
	\end{frame}

	\begin{frame}{Utilizando técnicas profundas}
	\begin{enumerate}
	\item Identificar candidatos na imagem.
	\item Introduzir a amostra em um classificador profundo, obtendo a classe correspondente.
	\end{enumerate}

	\fig{diagram/system-deep}{}[1.1][1]
	\end{frame}

	\begin{frame}{Panorama do trabalho}
		\begin{itemize}
			\item Detecção utilizando métodos tradicionais de aprendizado.
			\item Classificação utilizando aprendizado profundo.
			\item Resultados
		\end{itemize}
	\end{frame}
