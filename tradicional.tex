\section{Detecção utilizando métodos tradicionais}

\subsection{Obtenção de candidatos}
	\begin{frame}{\insertsubsection}
		\begin{itemize}
		\item<1-> Utilizar informação da profundidade.
		\item<2-> Hipótese: pessoas estão entre os objetos mais altos da cena.
		\item<2-> Solução: obter máximos locais.
		\end{itemize}

		\only<1>{
		\begin{columns}[T]
		\column{0.5\textwidth} \fig{tradicional/color}{Imagem de cor.}[1][0.7]
		\column{0.5\textwidth} \fig{tradicional/depth}{Imagem de profundidade.}[1][0.7]
		\end{columns}
		}
		\only<2->{\fig{tradicional/local_max}{}[1][0.7]}
	\end{frame}

	\begin{frame}{\insertsubsection}
		\begin{itemize}
		\item Estima-se quadrado sob o ponto de máximo.
		\item Utiliza-se \emph{mean shift} para centralizar o quadrado sob o candidato.
		\item O candidato é considerado como o recorte do quadrado sob a imagem original.
		\end{itemize}
		\fig{tradicional/candidates_depth}{Candidatos obtidos.}[1][0.5]
	\end{frame}

\subsection{Descritores de características}
	\begin{frame}{Grades simples}
		\begin{columns}
		\column{0.5\textwidth}
		\begin{enumerate}
			\item Dividir o candidato em 7x7 blocos iguais.
			\item Obter média dos pixels de cada bloco $\rightarrow$ Matriz de médias 7x7.
			\item Subtrair dessa matriz a média do bloco central.
			\item Gerar histograma da matriz resultante, $x \in \mathbb{R}^d$, com $d=32$ intervalos.
		\end{enumerate}
		\column{0.5\textwidth}
		\fig{tradicional/descGradesSimples}{}
		\end{columns}
	\end{frame}

	\begin{frame}{Anéis concêntricos}
		\begin{columns}
		\column{0.5\textwidth}
		\begin{enumerate}
			\item Dividir o candidato em $n$ coroas circulares com espaçamento uniforme.
			\item Obter vetor de médias dos pixels de cada coroa $\rightarrow$ $m \in \mathbb{R}^n$.
			\item Subtrair desse vetor a média da coroa interna, $m_0$.
			\item Diferenciar o vetor resultante, resultando em $x \in \mathbb{R}^d$  com $d=n-1$.
		\end{enumerate}
		\column{0.5\textwidth}
		\fig{tradicional/descAneis}{}
		\end{columns}
	\end{frame}

\subsection{Classificação}
	\begin{frame}{\textit{Support Vector Machine} binário}
		Seja o dataset o conjunto $(x_i, y_i)$ para $i=1 \dots N$ com $x_i \in \mathbb{R}^d$ e $y_i \in \{-1, 1\}$. Sejam os parâmetros do modelo $w \in \mathbb{R}^d$ e $b \in \mathbb{R}$.

		A função decisão é dada por
		\begin{equation*}
		f(x)=w^T x+ b.
		\end{equation*}

		\fig{svm/svm-margin}{}[1][0.6]
	\end{frame}

	\begin{frame}{Treinamento}
		O treinamento consiste na otimização (convexa) da função custo
		\begin{equation*}
			\label{eq:svm-cost}
			\min_{w,b} P(w,b) = \textcolor{blue}{\frac{1}{2}\|w\|^2} + C\textcolor{red}{\sum_i H_1(y_i f(x_i))},
		\end{equation*}

		\begin{itemize}
			\item<1-> Maximização da margem \emph{versus} erros de treinamento.
			\item<2-> Transformação de espaço: Kernel RBF, parâmetro $\sigma$.
			\item<3-> Escolha hiper-parâmetros: Validação Cruzada (5 conjuntos).
		\end{itemize}

		\only<1>{\fig{svm/svm-margin-ex}{}[1][0.5]}
		\onslide<2>{\fig{svm/svm-kernel}{}[1][0.5]}
	\end{frame}
