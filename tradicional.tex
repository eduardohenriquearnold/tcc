\section{Detecção utilizando métodos tradicionais}

\begin{frame}{Trabalho de referência}
	Michael Rauter. 2013. Reliable Human Detection and Tracking in Top-View Depth Images. In \textit{Proceedings of the 2013 IEEE Conference on Computer Vision and Pattern Recognition Workshops} (CVPRW '13).

	\begin{itemize}
		\item Detecção de candidatos baseados em imagem de profundidade.
		\item Descritores de características baseado em blocos simples.
		\item Classificador SVM.
		\item Ambiente controlado.
	\end{itemize}
\end{frame}

\begin{frame}{Imagens de profundidade}
\begin{columns}[T]
\column{0.5\textwidth} \fig{tradicional/color}{Imagem de cor.}[1][0.7]
\column{0.5\textwidth} \fig{tradicional/depth}{Imagem de profundidade.}[1][0.7]
\end{columns}
\end{frame}

\begin{frame}{Obtenção de candidatos}
	\begin{itemize}
	\item Hipótese: pessoas estão entre os objetos mais altos da cena.
	\item Solução: obter máximos locais.
	\item Centralização com \textit{mean shift}.
	\end{itemize}

	\fig{tradicional/candidates_depth}{}[1][0.7]
\end{frame}

\begin{frame}{Descritores de características}
	\begin{columns}[T]
	\column{0.5\textwidth}
		\center Grades simples
		\fig{tradicional/descGradesSimples}{}

	\column{0.5\textwidth}
		\center Aneis circulares
		\fig{tradicional/descAneis}{}
	\end{columns}
\end{frame}


\begin{frame}{Classificador \textit{Support Vector Machine} binário}
	\begin{columns}[T]
	\column{0.4\textwidth}
	A função decisão é dada por
	\begin{equation*}
	f(x)=w^T x+ b.
	\end{equation*}

	\fig{svm/svm-margin}{}[1][0.6]
	\column{0.6\textwidth}
	Otimização da função custo
	\begin{equation*}
		\label{eq:svm-cost}
		\min_{w,b} P(w,b) = \textcolor{blue}{\frac{1}{2}\|w\|^2} + C\textcolor{red}{\sum_i H_1(y_i f(x_i))},
	\end{equation*}

	\begin{itemize}
		\item Maximização da margem \emph{versus} erros de treinamento.
		\item Transformação de espaço: Kernel RBF, parâmetro $\sigma$.
		\item Escolha hiper-parâmetros: Validação Cruzada (5 conjuntos).
	\end{itemize}
	\end{columns}
\end{frame}
