\section{Detecção utilizando métodos tradicionais}

\begin{frame}{Obtenção de candidatos}
	\begin{itemize}
	\item<1-> Utilizar informação da profundidade.
	\item<2-> Hipótese: pessoas estão entre os objetos mais altos da cena.
	\item<2-> Solução: obter máximos locais.
	\item<2-> Centralização com \textit{mean shift}.
	\end{itemize}

	\only<1>{
	\begin{columns}[T]
	\column{0.5\textwidth} \fig{tradicional/color}{Imagem de cor.}[1][0.7]
	\column{0.5\textwidth} \fig{tradicional/depth}{Imagem de profundidade.}[1][0.7]
	\end{columns}
	}
	\only<2->{\fig{tradicional/candidates_depth}{}[1][0.7]}
\end{frame}

\begin{frame}{Descritores de características}
	\begin{columns}[T]
	\column{0.5\textwidth}
	\center Grades simples

	\fig{tradicional/descGradesSimples}{}

	\column{0.5\textwidth}
	\center Aneis circulares

	\fig{tradicional/descAneis}{}
	\end{columns}
\end{frame}


\begin{frame}{Classificador \textit{Support Vector Machine} binário}
	\begin{columns}[T]
	\column{0.4\textwidth}
	A função decisão é dada por
	\begin{equation*}
	f(x)=w^T x+ b.
	\end{equation*}

	\fig{svm/svm-margin}{}[1][0.6]
	\column{0.6\textwidth}
	Otimização da função custo
	\begin{equation*}
		\label{eq:svm-cost}
		\min_{w,b} P(w,b) = \textcolor{blue}{\frac{1}{2}\|w\|^2} + C\textcolor{red}{\sum_i H_1(y_i f(x_i))},
	\end{equation*}

	\begin{itemize}
		\item Maximização da margem \emph{versus} erros de treinamento.
		\item Transformação de espaço: Kernel RBF, parâmetro $\sigma$.
		\item Escolha hiper-parâmetros: Validação Cruzada (5 conjuntos).
	\end{itemize}
	\end{columns}
\end{frame}

