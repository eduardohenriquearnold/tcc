\thispagestyle{plain}

\medskip

\begin{center}
  \textbf{RESUMO}
\end{center}



\bigskip




O presente trabalho pretende elaborar um sistema de segurança industrial que requer detecção automática de pessoas. Esse sistema deve impedir que uma estrutura se movimente enquanto houver colaboradores sob uma área de risco. O foco do trabalho é na detecção automática de pessoas, que é feita com base em imagens de profundidade, visão computacional e algoritmos de aprendizado de máquina. São propostas duas soluções para o problema. A primeira se baseia em técnicas tradicionais de aprendizado de máquina utilizando extratores de características e classificador \textit{Support Vector Machine}. A segunda utiliza técnicas de aprendizado profundo, mais especificamente redes neurais artificiais. A análise de desempenho das soluções revelou que os métodos de aprendizado profundo apresentam desempenho superior ao das técnicas tradicionais. Além disso, observou-se que as técnicas de aprendizado profundo não se restringem a grandes conjuntos de dados (big data), mas que podem ser empregadas com sucesso em situações com volume moderado de amostras, inclusive desbalanceadas.

\textbf{Palavras-chave:} Visão computacional. Aprendizado de máquina. Classificação. Redes neurais artificiais. Aprendizado profundo.

\cleardoublepageempty

\thispagestyle{plain}

\begin{center}
	\textbf{ABSTRACT}
\end{center}

\bigskip

This work develops an industrial security system that requires automatic human detection. This system must block the movement of a certain structure while there are workers under a hazardous area. The focus is on human detection which is done based on depth images, computer vision and machine learning. Two solutions are presented. The first one is based on traditional learning techniques using feature extraction and a Support Vector Machine classifier. The second solution uses deep learning methods for classification, more specifically Convolutional Neural Networks. The performance analysis of both solutions revealed that the deep methods outperform traditional learning techniques. Moreover, it is shown that such deep learning methods are not restricted to big data applications, but could be used to moderate-sized, unbalanced datasets.

\textbf{Keywords:} Computer vision. Machine learning. Classification. Artificial neural networks. Deep learning.
