\documentclass[a5paper]{ufsc-thesis}  % escolha o tamanho do papel aqui
\usepackage{cmap}
\usepackage[utf8]{inputenc}
\usepackage[T1]{fontenc}
\usepackage{amsmath}

%Pacotes para imagens
\usepackage{graphicx}
\graphicspath{ {img/} }

% Preâmbulo
\titulo{Sistema de detecção de pessoas para aplicação de segurança em ambiente industrial}
\autor{Eduardo Henrique Arnold}
\data{\today}
\instituicao{Universidade Federal de Santa Catarina}
\local{Florianópolis, Santa Catarina-Brasil}
\tipotrabalho{Trabalho de Conclusão de Curso}
\orientador{Danilo Silva}
\programa{Programa de Graduação em Engenharia Elétrica}
\preambulo{Monografia submetida ao Programa de Graduação em Engenharia Elétrica da Universidade Federal de Santa Catarina como requisito para aprovação na disciplina EEL7890 – Trabalho de Conclusão de Curso (TCC).}
\centro{Centro Tecnológico -- CTC}
\assuntos{Visão Computacional,Inteligência Artificial, Aprendizado de Máquina}

\begin{document}
% Inicia parte pré-textual do documento capa, folha de rosto, folha de
% aprovação, aprovação, resumo, lista de tabelas, lista de figuras, etc.
\pretextual%
\imprimircapa%
\imprimirfolhaderosto*%
\clearpage
\imprimirfichacatalografica%
%\tableofcontents%
\textual%

\section{Introdução}

\subsection{Caracterização da aplicação}
	\begin{frame}{\insertsubsection}
	\fig{wanke3}{Indústria de eletrodomésticos com extrusoras de plástico.}
	\end{frame}

	\begin{frame}{\insertsubsection}
	\fig{wanke2}{Moldes das extrusoras precisam ser substituidos através de uma ponte rolante.}
	\end{frame}

	\begin{frame}{\insertsubsection}
	\fig{wanke1}{Estrutura da ponte rolante pela fábrica.}
	\end{frame}

	\begin{frame}{Sistema de segurança}
		\textbf{Objetivos} \\
		\begin{itemize}
			\item Detectar pessoas automaticamente na região de trabalho.
			\item Impedir movimentação da ponte ao detectar pessoas.
		\end{itemize}

		\pause

		\textbf{Funcionamento} \\
		\begin{itemize}
			\item Câmera de profundidade com vista superior da área de trabalho.
			\item Aprendizado de máquina e visão computacional para detecção de pessoas.
		\end{itemize}
	\end{frame}


\subsection{Métodos de detecção de objetos}
	\begin{frame}{Utilizando descritores}
	\begin{enumerate}
	\item Identificar candidatos.
	\item Utilizar um extrator de características para descrição do objeto.
	\item Introduzir a amostra, proveniente do descritor, em um classificador.
	\end{enumerate}

	\fig{diagram/system}{}[1.1][1]
	\end{frame}

	\begin{frame}{Utilizando aprendizado de representação}
	\begin{enumerate}
	\item Identificar candidatos na imagem.
	\item Introduzir a amostra em um classificador profundo, obtendo a classe correspondente.
	\end{enumerate}

	\fig{diagram/system-deep}{}[1.1][1]
	\end{frame}

	\begin{frame}{Panorama do trabalho}
		\begin{itemize}
			\item Detecção utilizando métodos tradicionais de aprendizado.
			\item Classificação utilizando aprendizado de representação.
			\item Resultados
		\end{itemize}
	\end{frame}


\chapter{Método tradicional}
\chapter{Classificação com aprendizado profundo}
\chapter{Localização e classificação com aprendizado profundo}
\chapter{Resultados}
\chapter{Conclusão}
\end{document}
