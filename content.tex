\section{Introduction}
Workers safety must be a priority in any industrial facility. There are certain areas that offer higher risk and thus should not be occupied during regular operation. An example is given by a home appliance factory that uses an overhead crane to shift iron molds towards the plastic extruding machines. These molds can be very heavy, offering risks to anyone working underneath the moving crane.

In this context it is helpful to have an automatic security system that detects humans on the path of the moving machine and causes it to halt in the event it finds a person. A video based solution would be ideal in this case, even considering that the factory environment is crowded with machines, molds and workers. Since the crane will be moving, the camera should be placed underneath it facing the factory floor from a height of 6m. This poses a challenge as background subtraction methods cannot be applied, requiring a more sophisticated detection algorithm.

Another challenge is that workers uniforms are not regular in color, and they do not always wear helmets, in which case only color images could not give enough information for detection. To overcome this, we use a stereo camera that provides depth image frames of objects up to 20m from the lens. This image will then be used for human detection since it provides more reliable shape information and higher invariance to luminosity.

This paper makes a comparison between two approaches to the human detection problem. Both uses computer vision techniques to detect candidates in the image. The first one, based on traditional computer vision with hand-engineered feature extraction and a binary SVM classifier, is presented in Section \ref{sec:classical}. The latter uses a deep learning based classifier and is presented in \ref{sec:deep}. Quantitative evaluation of both methods is presented in Section \ref{sec:results}, and finally, conclusions and future work are presented in Section \ref{sec:conclusion}.

\section{Classical computer vision approach}
\label{sec:classical}

\section{Deep learning based solution}
\label{sec:deep}

\section{Results}
\label{sec:results}

\section{Conclusion}
\label{sec:conclusion}
