\section{Resultados}

\subsection{Medidas de desempenho}
	\begin{frame}{\insertsubsection}
		Os classificadores utilizados possuem uma saída probabilística. É possível escolher um limiar de probabilidade $T$ acima do qual se considera a amostra como sendo da classe positiva (cabeça).

		Cada escolha de $T$ permite gerar uma \emph{matriz de confusão} que mostra a distribuição das amostras nas classes de \emph{verdadeiros positivos, falsos positivos, verdadeiros negativos e falsos negativos}.

		\begin{table}
		\centering
		\caption{Matriz de confusão}
		\label{tab:matriz-confusão}
		\begin{tabular}{ll|l|l|}
		\cline{3-4}
					                    &   & \multicolumn{2}{l|}{Classificado} \\ \cline{3-4} 
					                    &   & 0               & 1               \\ \hline
		\multicolumn{1}{|l|}{\multirow{2}{*}{Real}} & 0 & \text{TN}              & \text{FP}              \\ \cline{2-4} 
		\multicolumn{1}{|l|}{}                      & 1 & \text{FN}              & \text{TP}              \\ \hline
		\end{tabular}
		\end{table}	
	\end{frame}

	\begin{frame}{\insertsubsection}
		Cada matriz permite calcular a taxa de verdadeiro positivo
		\begin{equation}
		\text{TVP} = \frac{\text{TP}}{\text{FN}+\text{TP}}
		\end{equation}
		e a taxa de falso positivo
		\begin{equation}
		\text{TFP} = \frac{\text{FP}}{\text{TN}+\text{FP}}.
		\end{equation}

		 Assim, cada escolha de $T$ representa um ponto no espaço \emph{Receiver Operating Characteristic}. Variando $T$ é possível obter uma curva nesse espaço.

		A área sob essa curva \emph{Area Under Curve} (AUC) também pode ser utilizada como métrica para desempenho do sistema.
	\end{frame}


\subsection{Classificadores}
	\begin{frame}{Descritor anéis concêntricos}
		Avaliação do melhor número de anéis. Treinamento sob conjunto reduzido e avaliação no conjunto de testes.

		\fig{results/ROC_rings}{Curvas ROC dos classificadores usando diferentes descritores.}
	\end{frame}

	\begin{frame}{Todos os classificadores} 
		\only<1>{\fig{results/ROC_all}{Curvas ROC classificadores. Avaliação realizada no conjunto de testes.}[1][1]}
		\only<2>{\fig{results/ROC_all_zoom}{Curvas ROC classificadores zoom.}[1][1]}
	\end{frame}

\subsection{Sistema completo}
	\begin{frame}{\insertsubsection}
		Avalia-se agora o sistema completo: extração de candidatos e clasificador.

		Dado um quadro com múltiplos candidatos, cada cuja probabilidade de ser cabeça é $p_i$ para $i \in \{1 \dots n \}$, formula-se que a probabilidade de o quadro conter pelo menos uma cabeça é
		\begin{equation}
			P[y=1] = \max\{p_1, \ldots, p_n  \}.
		\end{equation}				
	\end{frame}

	\begin{frame}{\insertsubsection}
		\only<1>{\fig{results/ROC_system}{Curvas ROC sistema completo.}[1][0.9]}
		\only<2>{\fig{results/ROC_system_zoom}{Curvas ROC sistema completo zoom.}[1][0.9]}
		\only<3>{\fig{results/ROC_system_zoom_best}{Curvas ROC sistema completo zoom.}[1][0.9]}
	\end{frame}

	\begin{frame}{Vídeo de demonstração}
		Demonstração do sistema em funcionamento no vídeo de testes.

		\begin{center}
		\href{run:/usr/bin/mplayer ../img/demo.ogv}{Demo}
		\end{center}
	\end{frame}

