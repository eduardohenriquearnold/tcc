\section{Conclusão}

\subsection{}
\begin{frame}{\insertsection}
	\begin{itemize}
		\item Sistema de detecção de pessoas em um ambiente indústrial.
		\item Duas propostas de solução
		\begin{itemize}
			\item Métodos tradicionais de aprendizado
			\item Métodos de aprendizado profundo
		\end{itemize}

		\item Desempenho superior das técnicas profundas em relação aos métodos tradicionais de aprendizado.
		\item Bom desempenho mesmo com conjuntos medianos e desbalanceados.
	\end{itemize}
\end{frame}

\begin{frame}{Trabalhos futuros}
	\textbf{Ajuste fino do classificador}\\
	\begin{itemize}
		\item Video de testes limitado: poucos quadros sem pessoas, mesmos moldes.
		\item Avaliar a ocorrência de falsos positivos após instalação.
		\item Fazer treinamento de ajuste fino na camada de classificação.
	\end{itemize} 

	\pause

	\textbf{Proposta de solução integralmente profunda}\\
	\begin{itemize}
		\item Utizar uma rede neural convolucional.
		\item O quadro inteiro é tido como entrada do sistema.
		\item Saída bidimensional como mapa de probabilidades.
		\item Evita necessidade de seleção de candidatos.
	\end{itemize}
\end{frame}
