\section{Conclusão}

\subsection{}
\begin{frame}{\insertsection}
	\begin{itemize}
		\item Sistema de detecção de pessoas em um ambiente industrial.
		\item Duas propostas de solução
		\begin{itemize}
			\item Métodos tradicionais de aprendizado
			\item Métodos de aprendizado profundo
		\end{itemize}

		\item Desempenho superior das técnicas profundas em relação aos métodos tradicionais de aprendizado.
		\item Bom desempenho mesmo com conjuntos medianos e desbalanceados.
		\item Modelos profundos têm elevado custo computacional.
	\end{itemize}
\end{frame}

\begin{frame}{Trabalhos futuros}
	\textbf{Proposta de solução integralmente profunda}\\
	\begin{itemize}
		\item Utizar uma rede neural convolucional.
		\item O quadro inteiro é tido como entrada do sistema.
		\item Saída bidimensional como mapa de probabilidades.
		\item Evita necessidade de seleção de candidatos.
	\end{itemize}
\end{frame}

\begin{frame}{Bibliotecas utilizadas}
	\only<1>{\fig{libs/opencv}{}[1][0.5]}
	\only<2>{\fig{libs/scikit-learn}{}[1][0.5]}
	\only<3>{\fig{libs/theano}{}[1][0.5]}
	\only<4>{\fig{libs/keras}{}[1][0.5]}
\end{frame}
